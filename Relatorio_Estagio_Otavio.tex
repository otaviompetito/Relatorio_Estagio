\input{template_nota}
\usepackage{natbib}
\usepackage{tikz}
\graphicspath{ {./figs/} }
\tcbuselibrary{listings,breakable}

%Template da capa do documento

\title		{   
				\Huge
				Instituto Mauá de Tecnologia \\
				Escola de Engenharia Mauá \\
				\vspace{12px}
				\huge
				\textbf{
						Relatório de Estágio \\ 
						\Large 
						\vspace{12px}
						ETGSUP - Estágio Supervisionado Obrigatório \\
						}	
			}

\author		{
				Otávio Moreira Petito 08.01453-0 \\
			}
			
\titlepic	{\includegraphics[width=0.4\textwidth]{maua_logo}}

%Início do documento
\begin{document}

\maketitle
\tableofcontents

\chapter{Informações gerais}

\section{Informações do aluno}

	\begin{table} [hl]
	\begin{tabular}{ll}
		
		Nome: & Otávio Moreira Petito\\
		RA: & 08.01453-0\\
		Curso: & Engenharia Eletrônica\\
		Série: & 6ª\\
		Período: & Noturno\\
		Orientador: & Professor Dr. Wânderson de Oliveira Assis
		
	\end{tabular}
	\end{table}
	
\section{Informações da empresa}

	\includegraphics[width=0.2\textwidth]{siemens_logo}
	
	\begin{table} [hl]
	\begin{tabular}{ll}

		Empresa: & Siemens LTDA.\\
		Endereço: & Av. Mutinga, 3800\\
		Departamento: & RC-BR PD PA AE OEC ENG\\
		Supervisor: & Edgar Higashi\\
		CREA: & 5060544744\\
		
	\end{tabular}
	\end{table}
	
\chapter{Objetivo}

	O objetivo desse relatório é de identificar e descrever algumas das atividades realizadas pelo aluno e estagiário de Engenharia Elétrica - Ênfase Eletrônica, Otávio Moreira Petito, durante parte do período de estágio realizado na empresa Siemens LTDA, no período de Abril de 2014 até o presente momento.
	
\chapter{A empresa}

\section{Apresentação da empresa}

	A Siemens é atualmente o maior conglomerado de engenharia elétrica e eletrônica do país, com suas atividades agrupadas em quatro setores estratégicos – \textit{Industry}, \textit{Energy}, \textit{Healthcare} e \textit{Infrastructure \& Cities} – enquanto a Siemens \textit{IT Solutions and Services} atua nos quatro campos.

	Presente no Brasil há mais de cem anos, a Siemens tem sido uma empresa de tecnologia integrada que mantém posição de liderança em vários segmentos. A principal razão para esta permanência no topo está na enorme capacidade de adaptação às mudanças.

	Trabalhando para antever os desafios de amanhã e ajudar a sociedade a se preparar melhor, contemplando a qualidade de vida da população, o respeito ao meio ambiente, com a responsabilidade em criar condições econômicas que se traduzam em crescimento sustentável.(SIEMENS, 2015a)

\section{Breve histórico}
	
		A Siemens iniciou suas atividades no Brasil ainda no século 19, e fundou a subsidiária brasileira em 1905. Em mais de um século, assumiu a condição de maior empresa de tecnologia integrada do país. 

	O pioneirismo esteve presente na da companhia, sendo que a atuação mais antiga no Brasil aconteceu com a instalação da primeira linha telegráfica entre o Rio de Janeiro e o Rio Grande do Sul. É também responsável por outros marcos de inovação tendo instalado a primeira central telefônica automática da América Latina, em 1922, e a primeira fábrica de transformadores do Brasil, em 1939.

	Atualmente, os equipamentos e sistemas da empresa são responsáveis por 50\% da energia elétrica no país. No Brasil, o Grupo Siemens conta aproximadamente com dez mil colaboradores, seis centros de pesquisa, desenvolvimento e engenharia, treze unidades fabris e doze escritórios regionais de vendas e \textit{service}.(SIEMENS, 2015a)
	
\section{Ramo de atividades}

	Eletrificação, automação e digitalização são os campos de crescimento de longo prazo da Siemens. Para aproveitar totalmente o potencial de mercado nesses campos, os negócios são agrupados em nove divisões, além dos serviços de saúde, que constituem um negócio gerenciado separadamente.(SIEMENS, 2015b)
	
	{\it
	\begin{itemize}
		\item Power and Gas
		\item Wind Power and Renewables
		\item Power Generation Services
		\item Energy Management
		\item Building Technologies
		\item Mobility
		\item Digital Factory
		\item Power Industries and Drives
		\item Finantial Services
		\item Healthcare
	\end{itemize}}

\section{Linha de produtos}

	O setor de \textit{Power Industries and Drives} da Siemens oferece o que há de mais avançado em tecnologia de automação industrial, sistema de controle de processos, entre outros. Realizando todo o processo de desenvolvimento dos softwares destinados a essa finalidade.(SIEMENS, 2015c)
	
\section{Organograma}

	O organograma a seguir foi simplificada visando facilitar a localização do departamento de atuação.
	
		\includegraphics[width=0.6\textwidth]{Organograma}
	
\chapter{O estágio}	
	
\section{Descrição sucinta das atividades}

	A seguir serão apresentadas, brevemente, algumas das atividades desenvolvidas durante o estágio.

\subsection{Criação de EPH}

	Os EPHs (\textit{Equipments Phases}) são sequências de passos e transições, criados por SFC (\textit{Sequential Flow Charts}), que tem por objetivo fazer o controle dos módulos de equipamento, que são conjuntos de válvulas, sensores, bombas e etc. De forma resumida, pode-se dizer que o EPH é a inteligencia de um módulo de equipamento, pois garante que um conjunto de instruções sejam executadas em cada uma das fases de operação de um processo de fabricação. 
	
	Os \textit{Equipments Phases} foram criados utilizando o software da Siemens, \textbf{SIMATIC Manager}, baseado em um conjunto de instruções previamente descritos em fluxogramas e representados em documentos tipo P\&ID.
	
\subsection{Route Control System}

	Através de diagramas de tubulação e instrumentação, P\&IDs, de cada área de uma cervejaria, foram definidas as \textit{routes}, já contendo as \textit{partial routes} , \textit{locations} e os respectivos elementos (válvulas, bombas, entre outros) que seriam alocados. 
	
	Em uma macro em Excel foram criadas as \textit{locations} com as características indivíduais, que foram importadas no software \textbf{\textit{Route Control System}}, na sequência foram alocados os elementos, com as respectivas operações que poderiam ser assumidas. Com os elementos associados, foram realizadas as conexões das \textit{locations} gerando as \textit{partial routes}. E por fim, as conexões entre as \textit{partial routes} resultaram nas \textit{routes} para o processo de produção e CIP (\textit{clean-in-place}).

\subsection{Desenvolvimento e atualização de ferramentas de trabalho}

	Visando facilitar o andamento dos projetos e acompanhamento dos testes, foram desenvolvidas e/ou atualizadas macros em VBA para a realização de algumas tarefas, como a criação de planilhas de validação de testes, criação de banco de dados para desenvolvimento de software, além da elaboração de protocolos de teste para os projetos.

\subsection{Participação em TAF (Teste de Aceitação de Fábrica)}

	Durante o TAF (teste de aceitação de fábrica) o projeto encontra-se em fase final de desenvolvimento, iniciando os testes com o cliente para validação do trabalho realizado, podendo ser realizado na própria Siemens ou no cliente.
	
	No TAF, de uma indústria química, foi realizado no suporte ao desenvolvimento do projeto, executando atividades na criação e ajustes de SFCs.

\subsection{\textit{Upgrade} de sistema}

	A participação no \textit{upgrade} do sistema, de uma indústria química, contou com tarefas como a configuração de estações de trabalho, atualização de software do PCS7 versão 8.1, instalação e configuração do \textit{Process Historian} e \textit{Information Server}, ajustes de compatibilidade de receitas no BATCH e ajustes de telas de processo.

\chapter{Descrição detalhada de atividade}

	A seguir será apresentada, uma das atividades desenvolvida durante o estágio de forma detalhada.
	
\section{Desenvolvimento do software de simulação SIMIT com o simulador de campo SIMBA Profibus}

	O SIMIT \textit{Simulation Framework} é um \textit{software} de simulação que fornece ao usuário diversas funções e ferramentas necessárias em uma interface amigável para implementação de simulações das plantas de automação. O SIMBA Profibus é um \textit{hardware} capaz de simular a planta, se comunicando via Profibus com o CLP.
	
	Entre as vantagens da utilização do SIMIT, destacam-se a redução do tempo de comissionamento do projeto e a minimização de riscos durante fase de implementação e testes.
	
	\includegraphics[width=0.9\textwidth]{simit}
	
	O desenvolvimento do \textit{software} de simulação está diretamente ligado ao desenvolvimento do projeto no PCS7, pois é ele quem irá fornecer arquivos que facilitarão no andamento do projeto do SIMIT. 
	
	Ao iniciar o projeto no SIMIT, é necessário fazer a configuração do \textit{hardware} para depois fazer o \textit{download} dela no SIMBA Profibus. Para realizar tal tarefa, é necessário fazer no PCS7 a exportação dos arquivos do \textit{hardware config} e do \textit{symbol table} e importá-los no SIMIT.
	
	\includegraphics[width=1.0\textwidth]{hwconfig}
	
	Realizadas as importações, é necessário criar os 'típicos', que são os modelos para os elementos de campo, ou seja, válvulas \textit{on/off}, válvulas proporcionais, motores, entre outros.
	
	Para a criação dos típicos é necessário criar um novo \textit{template}.
	
	\begin{figure}[th]
	\centering
	\includegraphics[width=0.23\textwidth]{template}
	\end{figure}
	
	Após a criação do \textit{template} os elementos devem ser selecionados na biblioteca e arrastados para o \textit{chart}.
	
	\begin{figure}[th]
	\centering
	\includegraphics[width=0.95\textwidth]{tipico}
	\end{figure}
	
	Criados os típicos, pode ser feito o importe da hierarquia e dos elementos de processo. No PCS7 é exportado um arquivo xml contendo essas informações.
	
	\begin{figure}[th]
	\centering
	\includegraphics[width=0.95\textwidth]{xml}
	\end{figure}
	
	Os típicos criados no SIMIT devem possuir os mesmos nomes usados no típicos criados no PCS7, dessa forma o \textit{software} irá procurar o nome do típico no xml e associá-lo ao típico do SIMIT.
	
	\begin{figure}[th]
	\centering
	\includegraphics[width=0.9\textwidth]{hier_tip}
	\end{figure}\pagebreak
	
	Tendo a hierarquia criada e os elementos importados no SIMIT, parte-se para a criação dos \textit{flowcharts}. Nessa etapa são criadas os fluxos do processo baseados nos \textit{P\&ID} fornecidos pelo cliente.
	
	\begin{figure}[th]
	\centering
	\includegraphics[width=0.9\textwidth]{flowchart}
	\end{figure}\pagebreak
	
	Criados todos os \textit{flowcharts} necessários o \textit{software} está pronto para realizar as simulações. Para iniciar a simulação, basta clicar no botão de \textit{play} no menu superior. Quando o SIMIT entrar em \textit{runtime}, a interface do mesmo ficará laranja. A partir desse ponto, todo o processo pode ser simulado no sistema SCADA da Siemens, o WinCC. Conforme as válvula e/ou motores são acionados os elementos no SIMIT ficam azuis, indicando que o elemento está aberto ou ligado. Nos reatores também é possível observar algumas informações, como por exemplo o nível e a quantidade de material.
	
	\begin{figure}[th]
	\centering
	\includegraphics[width=1.0\textwidth]{runtime}
	\end{figure}
	
	O SIMIT é um \textit{software} extremamente útil para o desenvolvimento de projetos de automação industrial, a simulação é desenvolvida totalmente independente do \textit{software} de automação. Os testes que devem ser realizados para aceitação do \textit{software} são otimizados, reduzindo tempo de comissionamento e garantindo maior segurança durante a fase de testes.	
	
\chapter{Avaliação do estágio}

	O estágio realizado foi muito importante para a minha formação, tanto pelo conhecimento técnico adquirido como pela experiência profissional. As atividades realizadas contribuíram para uma visão mais abrangente com relação a automação industrial, desde os equipamentos e softwares utilizados, até sua concepção e implementação.
	
	Foi possível aplicar no trabalho conceitos vistos no curso de Engenharia Eletrônica, principalmente relacionados às disciplinas de Engenharia de Controle, Projetos de Sistemas de Controle, Instrumentação e Automação e Laboratório de Controle e Automação. Esse conhecimento prévio foi muito importante para uma melhor execução do trabalho, além de facilitar no aprendizado de novos conhecimentos durante o estágio. 
	
	O trabalho em equipe para o desenvolvimento dos projetos foi um fator muito importante durante o estágio, auxiliando no crescimento profissional. O desenvolvimento de um projeto em equipe possibilita a aquisição de novos conhecimentos, tendo em vista as experiências particulares de cada um.
	
	Acredito, portanto, que o estágio realizado foi de extrema importância para a minha formação profissional, me mostrando de forma abrangente essa área de atuação da engenharia.

\section{Assinaturas}

\begin{minipage}[t]{.6\textwidth}

	\rule{5cm}{0.1mm} \\
	Otávio Moreira Petito\\
	\tiny
	Estagiário
	
\end{minipage}
\begin{minipage}[t]{.6\textwidth}

	\rule{5cm}{0.1mm} \\
	Edgar Higashi\\
	\tiny
	CREA: 5060544744\\
	Supervisor de estágio
	
\end{minipage}
	
\chapter{Referências}

\begin{table} [hl]
\begin{tabular}{ll}


	SIEMENS, Sobre a Siemens, Website Siemens Brasil, 2015. Acesso em 30 de setembro de 2015.\\ Disponível em: http://w3.siemens.com.br/home/br/pt/cc/sobre-a-siemens/Pages/home.aspx\\\\
	
	SIEMENS, Businesses, Website Siemens Brasil, 2015. Acesso em 30 de setembro de 2015.\\ Disponível em: http://www.siemens.com/businesses/br/pt/\\\\
	
	SIEMENS, Automation, Website Siemens Brasil, 2015. Acesso em 30 de setembro de 2015.\\ Disponível em: http://w3.siemens.com.br/automation/br/pt/Pages/automacao.aspx

\end{tabular}
\end{table}

\end{document}